\documentclass[a4paper]{article}

\addtolength{\oddsidemargin}{-10pt}
\addtolength{\textwidth}{50pt}
\addtolength{\textheight}{70pt}
\addtolength{\voffset}{-40pt}

\title{Mobile antenna model description}
\author{Jan Šimbera}
\date{\today}

\usepackage{amsmath}
\usepackage{amssymb}
\usepackage[pdftex,unicode,hidelinks]{hyperref}
\usepackage[pdftex]{graphicx}

\begin{document}
\maketitle
\begin{abstract}
This document outlines a model that describes the spread of cell phone signal
for the purposes of mapping cell phone antenna coverage areas and detecting
the user's location somewhat more accurately.
\end{abstract}


\section{Model description} \label{sec:antmodel}

\subsection{Assumptions}
The connection of the cell phone to the mobile network is facilitated through
a system of antennas. Usually, the operator shares very little information
about these antennas; nevertheless, some parameters, such as antenna location,
are almost always available, sometimes through crowdsourced registers. We thus
assume the antenna locations $\vec{x_0} = (x_0,y_0)$ are fixed and known; we
also assume a planar coordinate system for simplicity.

\subsection{Signal spread model}\label{sec:sigmodel}
We assume three orthogonal effects on the signal strength $S_i(\vec{x})$
of antenna $i$ for a given location $\vec{x} = (x,y)$, that is:
\begin{itemize}
\item antenna power $P_i$,
\item distance from the antenna $d_i(\vec{x}) = ||\vec{x} - \vec{x_i}||$,
\item azimuth of the vector from the antenna to the location
    $\varphi_i(\vec{x}) = \arctan{\frac{y - y_i}{x - x_i}}$.
\end{itemize}

We can employ a probabilistic approach to the problem: we first model the
probability of a user being at a given location $\vec{x}$ if we know that he
or she is connected to an antenna $i$ and then use the Bayesian rule to
determine signal strength as the inverse conditional probability:
\begin{equation}
S_i(\vec{x}) = P(i|\vec{x}) = P(\vec{x}|i) \frac{P(i)}{P(\vec{x})}
\end{equation}
We can isolate the $P(i)$ as the antenna power $P_i$ and amalgamate the
remaining two terms to make the model agnostic to an apriori distribution of
users (i.e. population/settlement density):
\begin{equation}
S_i(\vec{x}) = P_i \frac{P(\vec{x}|i)}{P(\vec{x})} = P_i \cdot L_i(\vec{x})
\end{equation}
The resulting term $L_i(\vec{x})$ should capture the remaining two effects of
distance and azimuth. As we assume the effects to be orthogonal, we can
express it as
\begin{equation}
L_i(\vec{x}) = f_d(d_i(\vec{x})) \cdot f_a(\varphi_i(\vec{x}))
\end{equation}

\subsubsection{Distance decay}
As we operate on macroscopic scales, the distance decay of an EM signal is
governed by an inverse square law:
\begin{equation}
S_i(\vec{x}) \propto d_i(\vec{x})^{-2}
\end{equation}
This, however, would imply the signal strength goes to infinity near the
location of the antenna. Therefore, it is useful to employ a small correcting
factor. Using this, we arrive at a probabilistic description by a Cauchy
distribution, which is useful:
\begin{equation}
d_i(\vec{x}) \sim \mathrm{Cauchy}(0,\gamma_i)
\end{equation}
where $\gamma_i$ is the distance decay scale factor. We can use the probability
density function of the Cauchy distribution as our distance decay function:
\begin{equation}
f_d(d_i(\vec{x})) = \frac{1}{
    \pi \gamma_i \left[1 + \left( \frac{d_i(\vec{x})}{\gamma_i} \right)^2 \right]
} = \frac{\gamma_i}{\pi [d_i(\vec{x})^2 + \gamma_i^2]}
\end{equation}

\subsubsection{Angular distribution}
To model the angular dispersion of the signal, we can use the von Mises
distribution in a manner similar to the Cauchy distribution for distance decay.
The von Mises distribution is a circular analogue of the normal distribution:
\begin{equation}
f(\varphi) = \frac{\exp{[\kappa \cos{(\varphi - \alpha)}]}}{2\pi I_0(\kappa)}
\end{equation}
where $I_0(\kappa)$ is the modified Bessel function of order zero, $\alpha$ is
the principal angle and $\kappa$ is the directional concentration.

Using this, we define
\begin{equation}
f_a(\varphi_i(\vec{x})) = \frac
    {\exp{[\kappa_i \cos{(\varphi_i(\vec{x}) - \alpha_i)}]}}
    {2\pi I_0(\kappa_i)}
\end{equation}
which leaves us with two parameters for the antenna: orientation (principal
angle) $\alpha$ and angular signal concentration (narrowness) $\kappa$.

\subsubsection{Final signal strength equation}
Combining the above results, we arrive at the final signal model result
\begin{align}
S_i(\vec{x}) &= P_i \cdot f_d(d_i(\vec{x})) \cdot f_a(\varphi_i(\vec{x}))\\
&= P_i \frac
    {\gamma_i \exp{[\kappa_i \cos{(\varphi_i(\vec{x}) - \alpha_i)}]}}
    {2\pi^2 I_0(\kappa_i) [d_i(\vec{x})^2 + \gamma_i^2]}\\
&= \frac{P_i \gamma_i}{2\pi^2 I_0(\kappa_i)} \cdot \frac
    {\exp{[\kappa_i \cos{(\varphi_i(\vec{x}) - \alpha_i)}]}}
    {d_i(\vec{x})^2 + \gamma_i^2}\\
&= K_i \cdot \frac
    {\exp{[\kappa_i \cos{(\varphi_i(\vec{x}) - \alpha_i)}]}}
    {d_i(\vec{x})^2 + \gamma_i^2}
\end{align}

Since the signal strengths are relative to each other, we should specify a
condition on their absolute values; this can be formulated e.g. as
\begin{equation} \label{eq:powersumcond}
\sum_{a=1}^{n_a} P_a \gamma_a = 1
\end{equation}

\subsection{Antenna parameters}
The signal strength model uses four additional parameters (in addition to
location) to describe the antennas:
\begin{itemize}
\item Antenna orientation azimuth $\alpha \in [0;2\pi)$ (with zero pointing
    north).
\item Antenna power or overall signal strength $P \in \mathbb{R}^{+}$. This
    is somehow related to the power of the antenna transmitter.
\item Distance decay scale parameter $\gamma \in \mathbb{R}^{+}$, defining
    the rate of signal strength diminishing with increasing distance from the
    antenna. This can be related e.g. to the vertical orientation of the
    antenna with respect to the surface.
\item Angular concentration of antenna signal $\kappa \in \mathbb{R}^{+}_{0}$.
    The higher the concentration (narrowness), the more the signal power is
    concentrated along the direction of the antenna orientation. $\kappa = 0$
    means the antenna is isotropic and radiates the signal uniformly in all
    directions.
\end{itemize}
Sometimes, some of the parameters may be known a priori, such as the antenna
azimuths. The rest of the parameters need to be estimated from the antenna
connection data, which should be the next large task.



\section{Model parameter estimation}
To estimate the antenna parameters, several variants of data sources may be
used. In this estimation scenario, we consider a set of $n_p$ places $p$ with
locations $\vec{x}_p$. We denote the signal strength of a given antenna
$a$ at that place as $S_{ap}$, the distance $d_{ap}$ and angle $\varphi_{ap}$
of a given place from it as in \ref{sec:sigmodel}.

In addition to the place data, we enter data about a set of $n_u$ users and
their simultaneous connections to antennas (that is, how often they are
connected to what antennas while not moving themselves).
Therefore, we propose using night-time signalling (SS.7 logs) or CDR
data where antenna connection variability is most likely to come from antenna
signal strength variation, not from user movements. Given this, we may derive
for each user a set of connection time fractions $f_{ua}$ denoting the
likelihood of the connection of the user to the respective antennas, satisfying
$\sum_{a=1}^{n_a} f_{ua} = 1 \,\forall u$.

\subsection{Estimation with known locations} \label{sec:locest}
If we have an additional data source that allows us to determine the actual
night-time positions of the users (such as independent mobility tracker app
logs), the model estimation becomes simpler because we can compute the user
location probabilities with respect to the places $p$ directly as $p_{up}$.

The algorithm works as an EM procedure, trying to estimate the signal strengths
for all places and antennas with antenna parameters as proxy variables.
The E step estimates the antenna parameters from signal strengths using OLS,
with the initial values obtained from location averaging, and the M step
estimates the signal strengths from the antenna parameters using the equations
from \ref{sec:antmodel}.

\subsubsection{Antenna dominance fractions}
Dominance fractions denote how much the given place is dominated by a signal
from a given antenna, an therefore, the probability of a user at a given
location being connected to the antenna:
\begin{equation}
\Psi_{ap} = \frac
    {\sum_{u=1}^{n_u} p_{up} f_{ua}}
    {\sum_{u=1}^{n_u} p_{up}}
\end{equation}
This ensures that $\sum_{a=1}^{n_a} \Psi_{ap} = 1 \,\forall p$.

\subsubsection{Expectation step}
In this step, we compute the optimal antenna parameters using OLS on a
linearized signal strength function (with parameters from the previous round)
set as equal to the E-computed signal strength:
\begin{align}
\frac{1}{P_a^{(i)}} &\cdot P_a^{(i+1)} \nonumber \\
+ \frac
    {d_{ap}^2 - {\gamma_a^{(i)}}^2}
    {\gamma_a^{(i)} (d_{ap}^2 + {\gamma_a^{(i)}}^2)}
    &\cdot \gamma_a^{(i+1)} \nonumber \\
+ \kappa_a^{(i)} \sin{(\varphi_{ap} - \alpha_a^{(i)})}
    &\cdot \alpha_a^{(i+1)} \nonumber \\
+ \left[
        \cos{(\varphi_{ap} - \alpha_a^{(i)})}
        - \frac{I_1(\kappa_a^{(i)})}{I_0(\kappa_a^{(i)})}
    \right] &\cdot \kappa_a^{(i+1)} =\\
\frac{S_{ap}^{(i+1)}}{S_{ap}^{(i)}}
+ 1
+ \frac
    {d_{ap}^2 - {\gamma_a^{(i)}}^2}
    {d_{ap}^2 + {\gamma_a^{(i)}}^2}
+ \kappa_a^{(i)} \alpha_a^{(i)} \sin{(\varphi_{ap} - \alpha_a^{(i)})}
&+ \kappa_a^{(i)} \left[
        \cos{(\varphi_{ap} - \alpha_a^{(i)})}
        - \frac{I_1(\kappa_a^{(i)})}{I_0(\kappa_a^{(i)})}
    \right] \nonumber
\end{align}

\subsubsection{Maximization step}
In this step, we estimate the signal strengths for places from their antenna
dominances and antenna parameters:
\begin{equation}
S_{ap}^{(i+1)} = \Psi_{ap} \sum_{b=1}^{n_a} \frac
    {P_b^{(i)} \gamma_b^{(i)} \exp{[
        \kappa_b^{(i)} \cos{(\varphi_{bp} - \alpha_b^{(i)})}
    ]}}
    {2\pi^2 I_0(\kappa_b^{(i)}) [d_{bp}^2 + {\gamma_b^{(i)}}^2]}\\
\end{equation}

\subsubsection{Initial antenna parameter estimations}
The initial values can be obtained as follows:

\paragraph{Antenna principal angles} $\alpha_a$ -- if not known -- can be
determined as weighted circular means
\begin{equation}
\alpha_a = \arctan \frac
    {\sum_{p=1}^{n_p} \Psi_{ap} \sin \varphi_{ap}}
    {\sum_{p=1}^{n_p} \Psi_{ap} \cos \varphi_{ap}}
\end{equation}

\paragraph{Antenna angular concentrations} $\kappa_a$ can be determined
using the von Mises-Fisher iterative estimation procedure from the parameter
$\bar{R}_a$:
\begin{equation}
\bar{R}_a = \frac
    {\left(\sum_{p=1}^{n_p} \Psi_{ap} \sin \varphi_{ap}\right)^2
        + \left(\sum_{p=1}^{n_p} \Psi_{ap} \cos \varphi_{ap}\right)^2}
    {\left(\sum_{p=1}^{n_p} \Psi_{ap}\right)^2}
\end{equation}
With this parameter, we can produce the initial estimate for $\kappa_a$ as
\begin{equation}
\kappa_a^{(0)} = \frac
    {\bar{R}_a (2 - \bar{R}_a^2)}
    {1 - \bar{R}_a^2}
\end{equation}
and then repeat the following equation until convergence
\begin{equation}
\kappa_a^{(i+1)} = \kappa_a^{(i)} - \frac
    {A_p(\kappa_a^{(i)}) - \bar{R}_a}
    {1 - A_p(\kappa_a^{(i)})^2 - \frac{A_p(\kappa_a^{(i)})}{\kappa_a^{(i)}}}
\end{equation}
where $A_p(x) = \frac{I_1(x)}{I_0(x)}$ is the ratio of modified Bessel functions
of the first and zeroth order respectively. Three iterations are usually
sufficient.

\paragraph{Antenna distance decay parameter} $\gamma_a$ may be estimated as
the weighted mean distance of a dominated place from the antenna:
\begin{equation}
\gamma_a = \frac
    {\sum_{p=1}^{n_p} \Psi_{ap} d_{ap}}
    {\sum_{p=1}^{n_p} \Psi_{ap}}
\end{equation}

\paragraph{Strengths of the antennas} $P_a$, to honor the condition from
\eqref{eq:powersumcond}, can be estimated as the relative size of the antenna's
area of dominance to the total area of study. We need to factor out the
general influence of the distance decay parameter, too:
\begin{equation}
P_a = \frac
    {\sum_{p=1}^{n_p} \Psi_{ap} A_p}
    {\gamma_a \sum_{p=1}^{n_p} A_p}
\end{equation}
where $A_p$ is the area corresponding to place $p$ -- if the places are only
given as points, the areas may be computed e. g. using Voronoi polygons.


\subsection{Estimation with unknown locations}
If we do not have data about real user locations $p_{up}$, we can supply them by
employing an EM estimation procedure over the proposed approach.

To do this, it is useful to know at least roughly the expected user densities
over the places used $w_p$. These densities may be derived from an apriori
population distribution measure such as a census grid or, alternatively, a land
use layer such as GHSL aggregated to a reasonable level.

\subsubsection{Expectation step}
This step of the procedure estimates the user locations
from signal strengths in those places by computing place affinities as
\begin{equation}
a_{up}^{(i+1)} = \left(
    1 + \frac
        {w_p - \sum_{u=1}^{n_u} p_{up}^{(i)}}
        {w_p + \sum_{u=1}^{n_u} p_{up}^{(i)}}
\right)^{\frac{1}{k_c}}
\sqrt{
    \frac{1}{n_a} \sum_{a=1}^{n_a} \left(
        f_{ua} - \frac{S_{ap}^{(i)}}{\sum_{b=1}^{n_a} S_{bp}^{(i)}}
    \right)^2
}
\end{equation}
where $k_c$ is the crowding tolerance coefficient specifying the degree of
adherence to the apriori place weights $w_p$. Its higher values mean a more
different spatial distribution of users can arise.

Then the actual localization probabilities are produced by normalizing the
affinities to sum to one:
\begin{equation}
p_{up}^{(i+1)} = \frac
    {a_{up}^{(i+1)}}
    {\sum_{q=1}^{n_p} a_{uq}^{(i+1)}}
\end{equation}

\subsubsection{Maximization step}
This step of the procedure uses the method from
\ref{sec:locest} to derive $S_{ap}^{(i)}$ from $p_{up}^{(i)}$.
\end{document}